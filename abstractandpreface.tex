\subsection*{Abstract}
{\small In this bachelor's project, an interferometric and asteroseismic analysis of the subgiant star \mystar is presented. The aim of the project is to calculate the radius of this star using interferometry and then compare the result with a radius determined from asteroseismology. 

The interferometric analysis uses measurements from the CHARA array. A calibration procedure is performed and the limb-darkening coefficient is estimated using tables from stellar model atmospheres. The interferometric angular diameter is calculated from the fit to the measurements, and combined with the \hipparcos parallax, the linear radius of the star is found to be $R = \SI{2.07 \pm 0.04}{\solarradius}$. Using the interferometric angular diameter, the effective temperature of \mystar is determined to be $\teff = \SI{6211 \pm 91}{\kelvin}$.

The asteroseismic analysis uses short-cadence data from the \kepler mission. The determined asteroseismic parameters are used in one of the asteroseismic scaling relations to find the radius. The radius of \mystar is computed to be 
$R = \SI{2.20 \pm 0.05}{\solarradius}$.

The difference between the two radii is $\SI{2.03}{\sigma}$, which is significant.
It could be due to an assumption about the shape of the oscillation profile used in the asteroseismic analysis being wrong,
or it could be a systematic deviation of the scaling relation for this type of star. \par
}
\subsection*{Resumé}
\selectlanguage{danish} 
{\small Dette bachelorprojekt gennemgår en interferometrisk og asteroseismisk analyse af subkæmpestjernen \mystar. Formålet med projektet er at beregne stjernens radius ved brug af interferometri og sammenligne resultatet med en radius fundet ved asteroseismologi. 

Den interferometriske analyse bruger målinger foretaget på det sammenkoblede interferometer CHARA. Et kalibreringsprogram anvendes, og ud fra modeller af stjerneatmosfærer bestemmes randformørkelseskoefficienten. Den interferometriske vinkeldiameter findes fra et fit til målingerne, og når den kombineres med parallaksen målt af Hipparcos, bestemmes den lineære radius af stjernen til $R = \SI{2.07 \pm 0.04}{\solarradius}$. Ved at bruge vinkeldiameteren estimeres den effektive temperatur til at være $\teff = \SI{6211 \pm 91}{\kelvin}$.

Den asteroseismiske analyse bruger kort-kadence data fra \kepler-missionen. De herfra fundne asteroseismiske parametre bliver brugt i en af de asteroseismiske skalarelationer til at finde radius. Radius af \mystar findes til at være $R = \SI{2.20 \pm 0.05}{\solarradius}$. 

De to radier afviger med $\SI{2.03}{\sigma}$, hvilket er en betydelig forskel. Det kan skyldes, at antagelsen om formen af oscillationsprofilen brugt i den asteroseismiske analyse kan være forkert, eller det kan udtrykke, at resultater fra de asteroseismiske skalarelationer afviger systematisk for denne type stjerne. \par
}
\clearpage
\selectlanguage{english} 
\section*{Acknowledgements}
\vspace{-5pt}
Numerous people have contributed and helped with this project in one way or another, and I would like to thank all of them. I would like to thank my supervisors Victor~Silva~Aguirre and Tim~White for guiding me through the entire process and for helping me sculpt the project. I would also like to thank them for allowing me to continue to work with this star for the upcoming months in order for me to learn new and exciting methods.

I would like to thank Mathias Rav sincerely for providing me with his unfailing support and love, but also for his  valuable help and comments on this thesis.

I would also like to thank Jakob Rørsted Mosumgaard and Kenneth Lund Kjærgaard
for participating in many rich discussions and for giving helpful suggestions during this project, but also for sharing their interest in and enthusiasm for astronomy in the time we have known each other.


\clearpage