% Forside
\begin{titlingpage}
 \projecttitle
 \newpage



% Colophone
\clearpage
\newgeometry{left=5cm,right=2cm,bottom=2cm}
\thispagestyle{empty} % fjerne evt. sidehoved og -fod
\small
%resten af teksten indenfor dette env skal være \small
\strut\vfill % pres alt ned i bunden af siden
\begin{flushleft}
\projecttitle \par \projecttitledanish \par \vspace{11 pt}
Amalie Louise Stokholm, 2016 \par
Bachelor's project written under supervision of Victor~Silva~Aguirre and Timothy~White at Stellar~Astrophysics~Centre, Department~of~Physics~and~Astronomy, Aarhus~University \par\vspace{11 pt}
This document is typeset with \LaTeX\xspace using the \textsf{memoir} class. The text is set with \emph{Latin~Modern} in 11 point size. The choice of layout with the limit of 66 characters per line is quite deliberate since the author believes that the chosen layout improves readability, and with approximately 36 lines per page it adheres to the requirement that a page of text contains approximately 2400 characters. When the empty pages and extra white space are taken into account, the project is less than 30 pages. \par
\vspace{11 pt}
Printed at Aarhus~University \par\vspace{11 pt}
\emph{Cover page: The large upper figure shows an area of the sky in the constellation \emph{Cygnus}. The image is an infra-red picture constructed from three different bands, false coloured in three colours: J band (\SI{1.2}{\micro\meter}) coloured blue, H band (\SI{1.6}{\micro\meter}) coloured green, and K$_s$ band (\SI{2.2}{\micro\meter}) coloured red. This image is from the Two Micron All Sky Survey (2MASS) \citep{mass} found using the Aladin Sky Atlas \citep{aladin}. \\ The magnified part shows the star \mystar, which is the object of interest in this study.} 
\end{flushleft}
\end{titlingpage}
\restoregeometry